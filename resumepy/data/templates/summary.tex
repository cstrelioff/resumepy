\vskip0.5em
\begin{tikzpicture}[font=\normalsize, node distance=0em, outer sep=0em]
\tikzstyle{summ1}=[minimum height=2em, minimum width=7.5in, text width=7.5in, font=\bf, align=left]
\node[summ1] (summary1) {\Large Summary};
\end{tikzpicture}

\begin{tikzpicture}[font=\normalsize, node distance=0em, outer sep=0em]
\tikzstyle{summ2}=[minimum height=1em, minimum width=7.4in, text width=7.3in, align=left]
\node[summ2] (summary2) [below=of summary1] { %{{-summary-%}} };
\end{tikzpicture}
